\documentclass[12pt]{report}
\usepackage[utf8]{inputenc}
\usepackage[english, russian]{babel}
\usepackage{fontspec} 
\usepackage[14pt]{extsizes}
\usepackage{caption}
\usepackage{graphicx}
\usepackage[left=2cm, right=2cm, top=1cm, bottom=2.5cm]{geometry}
\usepackage{listings}
\usepackage{titlesec, blindtext, color} % подключаем нужные пакеты
\usepackage{setspace} % полуторный интервал
\usepackage{hyperref}
\definecolor{gray75}{gray}{0.75} % определяем цвет
\newcommand{\hsp}{\hspace{20pt}} % длина линии в 20pt
% titleformat определяет стиль
\titleformat{\chapter}[hang]{\Huge\bfseries}{\thechapter\hsp\textcolor{gray75}\hsp}{0pt}{\Huge\bfseries}
\defaultfontfeatures{Ligatures={TeX},Renderer=Basic} 
\setmainfont[Ligatures={TeX,Historic}]{Times New Roman}
\begin{document}
	\begin{titlepage}
		\centering
		{\scshape\LARGE МГТУ им. Баумана \par}
		\vspace{3cm}
		{\scshape\Large Лекции по анализу алгоритмов\par}
		\vfill
		\large \textit {Москва, 2020} \par
	\end{titlepage}
	\onehalfspacing
	\tableofcontents
	\chapter*{Лекция 1}
	\addcontentsline{toc}{chapter}{Лекция 1}
	\section*{Введение}
	\addcontentsline{toc}{section}{Введение}
	\begin{enumerate}
		\item\underline{\textbf{Оценки качества}}
		\begin{itemize}
			\item Универсальность
			\begin{itemize}
				\item Ресурсные
				\begin{itemize}
					\item трудоемкость
					\item память
				\end{itemize}
				\item ?
			\end{itemize}
			\item Проблемно-ориентированность
		\end{itemize}
	
		\item\underline{\textbf{Вход}}\par
		$D = \{d_{i} | i=1,m\}$
		
		\item\underline{\textbf{Длина входа}}\par
		$\mu_{Z}(D) = n, h = |D|$\par
		Пример:\par $|D| = 2n^{2} + 1$\par
		$\mu_{Z}(D) = n = \sqrt{\frac{|D| - 1}{2}}$
		
		\item\underline{\textbf{Трудоемкость}}\par
		$f_{A}(D) -$ число элементарных операций принятой модели вычислений, (выполненных механизмом реализаций по механизму A) заданных алгоритмом $A$ на входе $D$.\par
		$z\rightarrow A_{1}A_{2} \qquad f_{A_{1}}(D) < f_{A_{2}}(D)$\par 
		
		\item\underline{\textbf{$D_{n}$}}\par
		$D_{n} = \{D | \mu_{z}(D)=n\}$\par
		Пример:\par
		$\beta(d_{i})=16$\par
		$n=10$\par
		$|D_{10}|=2^{160}$
		
		\item\underline{\textbf{Лудший, худший и средний случаи}}\par
		$f_{A}^{\vee} \equiv^{def} min_{D \in D_{n}} f_{A}(D)$\qquad - худший\par
		$f_{A}^{\wedge} \equiv^{def} max_{D \in D_{n}} f_{A}(D)$\qquad - лучший\par
		$\bar{f_{A}}(n) = \sum_{D \in D_{n}}P(D)f_{A}(n) $\qquad - средний\par
		$P(D) -$ вероятность входа, $f_{A}(n) -$трудоемкость
		
		\item\underline{\textbf{Память}}\par
		$V(D) = V_{D} + V_{R} + V_{dop} + V_{EXE} + V_{stack}$\par
		$V_{D} = |D|$\par
		$V_{EXE}$ - часть ASM кода?\par
		$V_{A}^{\vee}, V_{A}^{\wedge}(n), \bar{V_{A}}(n) \rightarrow V_{dop}$\par
	\end{enumerate}

	\section*{Примеры анализа}
	\addcontentsline{toc}{section}{Примеры анализа}
	\begin{enumerate}
		\item\underline{\textbf{Объем}}\par
		$a \leftrightarrow b$\par
		\begin{itemize}
			\item $Swap(a, b)$\par
			$\qquad t \leftarrow a$\par
			$\qquad a \leftarrow b$\par
			$\qquad b \leftarrow t$\par
			$End$\par
			
			\item $Swap2(a,b)$\par
			$\qquad a \leftarrow a + b$\par
			$\qquad b \leftarrow a - b$\par
			$\qquad a \leftarrow a - b$\par
		\end{itemize}
		
		\item\underline{\textbf{for}}\par
		for $j \leftarrow 1$ to $n$\par
		$\qquad j \leftarrow 1$\par
		$\qquad$ code\par
		$\qquad j \leftarrow j + 1$\par
		$\qquad$ If $j <= n$, goto code\par
		
		\item\underline{\textbf{Sum}}\par
		$Sum(A, n; S)$\par
		$\qquad s \leftarrow 0$\par
		$\qquad$ for $k \leftarrow 1$ to $n$\par
		$\qquad\qquad s \leftarrow s + A[k]$\par
		$End$\par
		$f_{A}^{\vee}(n)=f_{A}^{\wedge}(n) = f_{A}(n) = 1 + 1 + n(3 + 3) = 6n + 2$
		
		\item\underline{\textbf{MultMatr}}\par
		$MultMatr(A, B, n; C)$\par
		$\qquad$ for $i \leftarrow 1$ to $n$\par
		$\qquad\qquad$ for $ j \leftarrow 1$ to $n$\par
		$\qquad\qquad\qquad s \leftarrow 0$\par
		$\qquad\qquad\qquad$ for $k \leftarrow 1$ to $n$\par
		$\qquad\qquad\qquad\qquad s \leftarrow s + A[i,k]*B[k,j]$\par
		$\qquad\qquad\qquad C[i,j] \leftarrow s$\par
		$End$\par
		$f_{A}(n) = 1 + n(3 + 1 + n(3 + 1 + 1 + n(3 + 7) + 3)) = 10n^{3} + 8n^{2} + 4n + 1$\par
	\end{enumerate}

	\chapter*{Лекция 2}
	\addcontentsline{toc}{chapter}{Лекция 2}
	\section*{Примеры анализа}
	\addcontentsline{toc}{section}{Примеры анализа}
	\begin{enumerate}
		\item \underline{\textbf{Умножение матриц}} $\rightarrow f_{A}(D) = f_{A}(n)$
		\item \underline{\textbf{Возведение в степень}} $y = x^{k}$\par
		$Pow(x,k;y)$\par
		\underline{! Проверки} $x, k | k \geq 1$\par
		$y \leftarrow 1$\par
		for $j \leftarrow 1$ to $k$\par
		$\qquad y \leftarrow y + x$\par
		$End$\par
		\underline{? Вопрос}
		\begin{enumerate}
			\item $\mu_{z}(D)=2$
			\item $f_{A}(D)=f_{A}(k)=2+k^{1}$
		\end{enumerate}
		\item \underline{\textbf{Max$(A, n; M)$}}\par
		$M \leftarrow A[1]$\par
		for $j \leftarrow 2$ to $n$\par
		$\qquad$ if $M < A[j]$\par
		$\qquad\qquad$then\par
		$\qquad\qquad M \leftarrow A[j]$\par
		$End$\par
		\underline{Лучший, худший случаи:}\par
		$f_{A}^{\vee}(n)=^{Max=a_{1}} 3 + (n-1)(3+2)=5n-2$\par
		$f_{A}^{\wedge}(n)=3+(n-1)(3+2+2)=7n-4$\par
	\end{enumerate}

	\section*{Классификация \underline{алгоритмов} по типу трудоемкости}
	\addcontentsline{toc}{section}{Классификация алгоритмов по типу трудоемкости}	
	\begin{enumerate}
		\item \underline{\textbf{Класс $N$}}\par
		Количественно-зависимые\par
		$f_{A}(D)=f_{A}(\mu_{z}(D))=f_{A}(n)$\par\par
		Матрично-векторные операции\par
		\item \underline{\textbf{Класс $PR$}}\par
		Параметр-зависимые\par
		$f_{A}(D)=f_{A}(pr_{1},pr_{2},...,pr_{k})$\par\par
		Стандартные функции по $eps$\par
		\item \underline{\textbf{Класс $NPR$}}\par
		Количественно-параметрические алгоритмы\par
		$f_{A}(D)=f_{A}(n,pr_{1},...)$\par
		Логическая конструкция\par
		$If ...$\par
		$\qquad then$\par
		$\qquad\qquad ...$\par
		$\qquad else$\par
		$\qquad\qquad ...$\par
		\item \underline{\textbf{Декомпозиция $f_{A}(D)$}}\par
		$ln(x_{1}+x_{2})$\par
		\begin{enumerate}
			\item Аддитивно\par
			$f_{A}(n,pr)=f_{n}(n)+g(pr)$\par
			$f_{A}^{\wedge}(n)=f_{n}(n)+g^{\wedge}(n)$\par
			$g^{\wedge}(n)$ - $max_{D \in D_{n}}(g(pr))$\par
			\item Мультипликативность\par
			$f_{A}(D)=f_{n}(n)*g(pr)$\par
		\end{enumerate}
		\item \underline{\textbf{$\prec$ (П.Д.Раймон)}}\par
		$a<b\qquad f(.)\prec g(.)\qquad f(x)$\par
		$f\prec g\equiv^{def} lim_{x\rightarrow \infty} \frac{f}{g}=0$\par
		$f \asymp g \equiv^{def}lim...=C\neq 0$\par
		\item \underline{\textbf{Подклассы в $NPR$}}
		\begin{enumerate}
			\item $ NPRL(Low)$\par
			$g^{\wedge}(n) \prec f_{n}(n)$\par
			$f_{n}(n)=17n^{4}+...$
			\item $NPRE(Eq...)$\par
			$g^{\wedge}(n)\asymp f_{n}(n)$
			\item $NPRH(H...)$\par
			$g^{\wedge}\prec f_{n}(N)$\par
		\end{enumerate}
	\end{enumerate}	

	\chapter*{Лекция 3}
	\addcontentsline{toc}{chapter}{Лекция 3}
	\section*{Математические сведения}
	\addcontentsline{toc}{section}{Математические сведения}
	\begin{enumerate}
		\item \textbf{\underline{Принцип Дирихле}}\par
		P.D.L.Dirichle \par
		$A, B: |A| > |B|, f:F A\rightarrow B\Rightarrow f^{-1}\notin F$
		\item \textbf{\underline{Отношение}}\par
		$A: R \subseteq A x A$\par
		Если \begin{enumerate}
			\item $(a,a)\in R$ - рефлексивно
			\item $(a_1,a_2)\in R \Rightarrow$ - симметрично
			\item $(a_1,a_2)\in R$ и $(a_2,a_3)\Rightarrow (a_1,a_2\in R)$ - транзитивно
		\end{enumerate}\par
		$a + b + c\rightarrow R$, отношение эквивалентно.
		\item \textbf{\underline{Теория вероятностей}}\par
		$M_0 = <\Omega , p(.),\sigma_{\Omega}>$\par
		$p(0)=0,p(\Omega)=1$\par
		Случайная величина $X:\Omega_0 \rightarrow R^1$\par
		$r = X(\omega)$\par
		$P(X(\omega)=r^x)=\sum_{\omega :X(\omega)=r^x}P(\omega)$\par
		Математическое ожидание: $E(x)\equiv^{def} \sum_{i=1}^{M}x_i*P(X=x_i)$\par

	\end{enumerate}

	\section*{Метод классов эквивалентности}
	\addcontentsline{toc}{section}{Метод классов эквивалентности}
	\begin{enumerate}
		\item \textbf{\underline{Идея}}\par
		$NPR,PR$\par
		$\Rightarrow M_0 = <\Omega_0=D_n, P(.):P(d)=\frac{1}{|D_n|}>$\par
		Алгоритм $A\rightarrow f_A(D)$ - число\par
		$f_A(.): D_n \rightarrow_{f_A(.)}N$
	\end{enumerate}

	\chapter*{Лекция 4}
	\addcontentsline{toc}{chapter}{Лекция 4}
	\section*{Поиск $max$ - $\bar{f}_A(n)$}
	\addcontentsline{toc}{section}{Поиск $max$ - $f_A(n)$}
	\begin{enumerate}
		\item \textbf{\underline{Декомпозиция $f_A$}}\par
		$Max(A,n;M)$\par
		$\qquad M \leftarrow A[1]$\par
		$\qquad $for $ j \leftarrow 2 $ to $ n$\par
		$\qquad\qquad $If $ M<A[j]$\par
		$\qquad\qquad $then\par
		$\qquad\qquad\qquad M\leftarrow A[j]$\par
		End\par

		!Класс NPR
		$f_A(D\in D_n)=f_n(n)+g(D)$\par
		$f_n(n)=1+(n-1)(3+2)=5n-4$\par
		$g(D)=2Y_n$ ($Y_n$ - сл вел = числу присв Max)\par
		$Y_n \in [1,n] \Rightarrow \bar{f_A}(n)=E(f_n(n)+2Y_n)=f_n(n)+2E(Y_n)$

		\item \textbf{\underline{Применение метода классов}}\par
		$n=n_0\rightarrow $ экстраполяция на произвольную $n$\par
		$n_0 = 4$\par

		$D \in D_4 \rightarrow "1"(min),"2"(min_2),"3"(max_2), "4"(max)$\par
		$D_1234,...,D_2431,...,D_4321 \}= 24$\par
		$4xxx \}G \Rightarrow Y_4=1$\par
		$3xxx \}G \Rightarrow Y_4=2$\par
		$2134 \rightarrow 3 \qquad 1234 \rightarrow 4$\par
		$2143 \rightarrow 2 \qquad 1243 \rightarrow 3$\par
		$2314 \rightarrow 3 \qquad 1324 \rightarrow 3$\par
		$2341 \rightarrow 3 \qquad 1342 \rightarrow 3$\par
		$2413 \rightarrow 2 \qquad 1423 \rightarrow 2$\par
		$2431 \rightarrow 2 \qquad 1432 \rightarrow 2$\par
		$P(Y_4=1) = \frac{6}{24}$\par
		$P(Y_4=2) = \frac{11}{24}$\par
		$P(Y_4=3) = \frac{6}{24}$\par
		$P(Y_n=4) = \frac{1}{24}$\par
		$E(Y_n)=1\frac{6}{24} + 2\frac{11}{24} + 3\frac{6}{24} + 4\frac{1}{24}$\par
		Р.грэхем\par
		$x^{\bar{k}} = h_k(x)=x(x+1)(x+2)...(x+k-1)$\par
		$x^{\bar{4}} = x(x+1)(x+2)(x+3) = 1x^4+6x^3+11x^2+6x$\par
		$h_4(1)=4!=24$\par
		$E(Y_n)=\sum_{k=1}^nk\frac{S_n^{(k)}}{n!}$\par

		\item \textbf{\underline{Метод математических ожиданий}}\par
		Т.В. Теорема\par
		Если $Y=\sum_{i=1}^Mx_i$ и $E(x_i)$ - существует, то\par
		$E(Y)=E(\sum x_i) = \sum (E(x_i))$\par
		Из любого сходящегося ряда можно сделать распределение\par
		$!\sum_{k=1}^\infty \frac{1}{k^2}=\zeta (z)=\frac{\pi^2}{6}$\par
		$P(x=k)=\frac{6}{\pi^2}\frac{1}{k^2}$\par
		$E(x)=\sum^{\infty}k\frac{6}{\pi^2}\frac{1}{k^2} = \frac{6}{\pi^2}\sum \frac{1}{k^2}$\par
		$!x_j\rightarrow$ число переприсваиваний на $j$-ом шаге\par
		$X_j=\left\{
			\begin{array}{c}
				0\\
				1
			\end{array}
		\right.$\par
		$Y_n=\sum_{j=1}^n x_j$\par
		$j=1 = \left\{
			\begin{array}{c}
				x_1=0,\qquad P(.)=0\\
				x_1=1,\qquad P()=1
			\end{array}
		\right.$\par

		$E(x_k)=0(1-\frac{1}{k}) + 1\frac{1}{k_n}=\frac{1}{k}$\par
		$E(Y_n)=\sum_{k=1}^n E(x_k)=\sum_{k=1}^n \frac{1}{k}=H_n \approx \ln n + \gamma$\par
		$\bar{f}_{A_{max}} = 5n-4+2\ln n+2\gamma$
	\end{enumerate}


	\chapter*{Лекция 5}
	\addcontentsline{toc}{chapter}{Лекция 5}
	\section*{Сортировка вставками}
	\addcontentsline{toc}{section}{Сортировка вставками}
	\begin{enumerate}
		\item \textbf{\underline{Идея}}\\
		Ключ key и сдвигается до нужного
		\item \textbf{\underline{Текст}}\\
		$SortIns(A,n)$\par
		$\qquad A[0] \leftarrow MinType$\par
		$\qquad$ for $i \leftarrow 2$ to $n$\par
		$\qquad\qquad $key$ \leftarrow A[i]$\par
		$\qquad\qquad j \leftarrow i-1$\par
		$\qquad\qquad$ while$(A[j] > k)$\par
		$\qquad\qquad\qquad A[j+1] \leftarrow A[j]$\par
		$\qquad\qquad\qquad j \leftarrow j-1$\par
		$\qquad\qquad A[j+1] \leftarrow$ key\par
		$\qquad$End\par
		\item \textbf{\underline{Анализ $\bar{f_A}(n)$}} (класс NPR)\par
		$f_A(D \in D_n) = f_n(n) + g_{Pr}(D) = f_n(n) + 8Y_n$\par
		$Y_n$ - общее число проходов while\par
		$f_n(n)=2+1+(n-1)(3+4+2+3) = 12n - 9$\par
		$Y_n \in [0,?]$\par
		\item \textbf{\underline{$\bar{f_A}(n)$}}\\
		$\bar{f_A}(n) = E(f_A(D \in D_n)) = E(f_n(n) + 8Y_n) = f_n(n)+8E(Y_n)$\\
		$?E(Y_n) = \sum_{k=0}^{?}kP(Y_n=k)$(не подходит)\\
		\underline{Метод матожиданий} $\rightarrow Y_n = \sum_{2}^{n} X_i$, где $X_i$ - число проходов при $i = i_{for}$\\
		$E(X_i) \rightarrow -\circ -\circ -\circ ... \circ - |_{0}^i$\\
		($-\circ -\circ -\circ -\circ |_{0}^{i=4}$)\\
		$X_i = (i-1)$\\
		$X_i \in [0, i-1]$ $P(X_i = k)=\frac{1}{j}$\\
		$\Rightarrow E(X_i)=\sum_{k=0}^{i-1}k\frac{1}{i} = \frac{1}{i}(0+1+2+...+(i-2)+(i-1))=\frac{i1}{2}$\\
		$E(Y_n)=E(\sum X_i)=\sum_{2}^{n}(E(X_i))=\sum_{i=2}^{n}\frac{i-1}{2} = \frac{1}{2}\sum_{i=1}^{n}i - \frac{1}{2} -\frac{1}{2}(n-1)=\frac{n^2}{4}-\frac{n}{4}$\\
		$\bar{f_A}(n)=12n-9+8(\frac{n^2}{4}-\frac{n}{4}) = 2n^2 + 10n - 9$
	\end{enumerate}

	\section*{Сортировка индексами}
	\addcontentsline{toc}{section}{Сортировка индексами}
	\begin{enumerate}
		\item \textbf{\underline{Дано}}
		\begin{enumerate}
			\item $A \rightarrow a_i \in N$
			\item $a_i \neq a_j, i \neq j$
		\end{enumerate}
		\item \textbf{\underline{Идея}}\\
		$B[1..Max(a_i)]$\\
		$A=[7,11,3,5], B=[0,0,1,-,1,-,1,-,-,-,1]$\\
		$!B[A[i]]\leftarrow 1$\\
		Используем элемент массива $A$ как элемент массива $B$\\
		Цикл по $n$
		\item \textbf{\underline{Сборка без if}}\\
		$k \leftarrow 1$\par
		for $j \leftarrow 1$ to $Max$\par
		$\qquad A[k] \leftarrow j$\par
		$\qquad k \leftarrow k + B[j]$\par
		End\\
		Сборка без if - отвратительная\\
		\textbf{\underline{Модификации}}\\
		Уберем (b) $\rightarrow \leftarrow 1 \Rightarrow +1$\\
		(a) $N \rightarrow Z$
		\item \textbf{\underline{Текст (с if)}}\\
		$Sort(A, n, B, min, max)$\par
		$\qquad L \leftarrow Max-Min+1$\par
		$\qquad$for $j \leftarrow 1 to L$\par
		$\qquad\qquad B[j] \leftarrow 0$\par
		$\qquad$for $i \leftarrow 1$ to $n$\par
		$\qquad\qquad k \leftarrow A[i] - Min + 1$\par
		$\qquad\qquad B[k] \leftarrow B[k] + 1$\par
		$\qquad k \leftarrow 1$\par
		$\qquad$for $j \leftarrow 1$ to $L$\par
		$\qquad\qquad$ If $B[j] \neq 0$\par
		$\qquad\qquad\qquad$ then\par
		$\qquad\qquad\qquad m \leftarrow B[j]$\par
		$\qquad\qquad\qquad$for $s \leftarrow 1$ to $m$\par
		$\qquad\qquad\qquad\qquad A[k] \leftarrow j + Min - 1$\par
		$\qquad\qquad\qquad\qquad k \leftarrow k + 1$\par
		$\qquad\qquad\qquad$ End\par
		$\qquad$ End\par
		End.\par
		
		\underline{Анализ при $m=1 \Rightarrow f^{\wedge}$}\\
		$f_A(n, L) = 5L + 11n + 5L + n(3 + 3 + 6) + C = 10L + 23n + C$\par
		
		\item \textbf{\underline{Сравнение}}
		$\bar{f_{A_1}}(n) = 2n^2 + 10n$\\
		$f_{A_2}(n,L) = 10L + 23n$\\
		$L = \frac{2n^2 - 13n}{10} = \frac{1}{5}n^2 - 1,3n$
	\end{enumerate}
	
	\chapter*{Лекция 6}
	\addcontentsline{toc}{chapter}{Лекция 6}
	\section*{Теория сложности (вычислений)}
	\addcontentsline{toc}{section}{Теория сложности (вычислений)}
	\begin{enumerate}
		\item \textbf{\underline{Обозначения в асимптотическом анализа функций}}\par
		\begin{enumerate}
			\item $\delta(.)$ - К. Бахман (кон XIX)\\
			$f(x)=\delta(g(x))$ $\exists c > 0, x_0:$\\
			$\forall x > x_0$ $0 \leq f(x) \leq g(x) C$\\
			\item $\Omega$ (Д.Э.Киут 1961г.)\\
			\item $\theta$ (Киут)\\
			$C_1 g(1 \leq f(x) \leq c_2 g())$\\
		\end{enumerate}
	\end{enumerate}




\end{document}