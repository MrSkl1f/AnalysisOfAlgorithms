\documentclass[12pt]{report}
\usepackage[utf8]{inputenc}
\usepackage[english, russian]{babel}
\usepackage{fontspec} 
\usepackage[14pt]{extsizes}
\usepackage{caption}
\usepackage{graphicx}
\usepackage[left=2cm, right=2cm, top=2cm, bottom=2.5cm]{geometry}
\usepackage{listings}
\usepackage{titlesec, blindtext, color} % подключаем нужные пакеты
\usepackage{setspace} % полуторный интервал
\usepackage{hyperref}
\definecolor{gray75}{gray}{0.75} % определяем цвет
\newcommand{\hsp}{\hspace{20pt}} % длина линии в 20pt
% titleformat определяет стиль
\titleformat{\chapter}[hang]{\Huge\bfseries}{\thechapter\hsp\textcolor{gray75}\hsp}{0pt}{\Huge\bfseries}

% Для листинга кода:
\lstset{ 
	language=[Sharp]C,                 % выбор языка для подсветки (здесь это С)
	basicstyle=\small\sffamily, % размер и начертание шрифта для подсветки кода
	numbers=left,               % где поставить нумерацию строк (слева\справа)
	numberstyle=\tiny,           % размер шрифта для номеров строк
	stepnumber=1,                   % размер шага между двумя номерами строк
	numbersep=5pt,                % как далеко отстоят номера строк от подсвечиваемого кода
	showspaces=false,            % показывать или нет пробелы специальными отступами
	showstringspaces=false,      % показывать или нет пробелы в строках
	showtabs=false,             % показывать или нет табуляцию в строках
	frame=single,              % рисовать рамку вокруг кода
	tabsize=2,                 % размер табуляции по умолчанию равен 2 пробелам
	captionpos=t,              % позиция заголовка вверху [t] или внизу [b] 
	breaklines=true,           % автоматически переносить строки (да\нет)
	breakatwhitespace=false, % переносить строки только если есть пробел
}
\defaultfontfeatures{Ligatures={TeX},Renderer=Basic} 
\setmainfont[Ligatures={TeX,Historic}]{Times New Roman}
\begin{document}
	
	%\def\chaptername{} % убирает "Глава"
	\begin{titlepage}
		\begin{table}[ht]
			\centering
			\begin{tabular}{|c|p{420pt}|} 
				\hline
				\begin{tabular}[c]{@{}c@{}} \includegraphics[scale=0.37]{source/EmblemBMSTU} \\\end{tabular} &
				\footnotesize\begin{tabular}[c]{@{}c@{}}\textbf{Министерство~науки~и~высшего~образования~Российской~Федерации}\\\textbf{Федеральное~государственное~бюджетное~образовательное~учреждение}\\\textbf{~высшего~образования}\\\textbf{«Московский~государственный~технический~университет}\\\textbf{имени~Н.Э.~Баумана}\\\textbf{(национальный~исследовательский~университет)»}\\\textbf{(МГТУ~им.~Н.Э.~Баумана)}\\\end{tabular}  \\
				\hline
			\end{tabular}
		\end{table}
		\noindent\rule{\textwidth}{4pt}
		\noindent\rule[14pt]{\textwidth}{1pt}
		\hfill 
		\noindent
		\makebox{ФАКУЛЬТЕТ~}%
		\makebox[\textwidth][l]{\underline{~~~~«Информатика и системы управления»~~~~~~~~~~~~~~~~~~~~~~~~~~~~~~~~~~~~~~~~~~~~}}%
		\\
		\noindent
		\makebox{КАФЕДРА~}%
		\makebox[\textwidth][l]{\underline{~~~~~~~«Программное обеспечение ЭВМ и информационные технологии»~~~~~~~~}}%
		\\
		
		
		\begin{center}
			\vspace{3cm}
			{\bf\huge Отчёт\par}
			{\bf\Large по лабораторной работе № 2\par}
			\vspace{0.5cm}
		\end{center}
		
		
		\noindent
		\makebox{\large{\bf Название:}~~~}
		\makebox[\textwidth][l]{\large\underline{~Алгоритмы умножения матриц~}}\\
		
		\noindent
		\makebox{\large{\bf Дисциплина:}~~~}
		\makebox[\textwidth][l]{\large\underline{~Анализ алгоритмов~}}\\
		
		\vspace{1.5cm}
		\noindent
		\begin{tabular}{l c c c c c}
			Студент      & ~ИУ7-55Б~               & \hspace{3.5cm} & \hspace{3.5cm}                 & &  Д.О. Склифасовский \\\cline{2-2}\cline{4-4} \cline{6-6} 
			\hspace{3cm} & {\footnotesize(Группа)} &                & {\footnotesize(Подпись, дата)} & & {\footnotesize(И.О. Фамилия)}
		\end{tabular}
		
		\vspace{1cm}
		
		\noindent
		\begin{tabular}{l c c c c}
			Преподователь & \hspace{6cm}   & \hspace{3.5cm}                 & & ~~~~~~ Л.Л. Волкова ~~~~~~\\\cline{3-3} \cline{5-5} 
			\hspace{3cm}  &                & {\footnotesize(Подпись, дата)} & & {\footnotesize(И.О. Фамилия)}
		\end{tabular}
		
		\begin{center}	
			\vfill
			\large \textit {Москва, 2020}
		\end{center}
		
		\thispagestyle {empty}
		\pagebreak
	\end{titlepage}
	\restoregeometry
	
	\tableofcontents
	\onehalfspacing
	\newpage
	\chapter*{Введение}
	\addcontentsline{toc}{chapter}{Введение}
	Цель работы: изучение алгоритмов умножения матриц. В данной лабораторной работе рассматриваются 3 алгоритма:
	\begin{enumerate}
		\item[1)] стандартный алгоритм умножения матриц;
		\item[2)] алгоритм Винограда;
		\item[3)] модифицированный алгоритм Винограда.
	\end{enumerate}
	\noindent Также требуется изучить рассчет сложности алгоритмов.
	В ходе лабораторной работы необходимо:
	\begin{enumerate}
		\item[1)] изучить алгоритмы умножения матриц;
		\item[2)] оптимизировать алгоритм Винограда;
		\item[3)] дать теоритическую оценку стандартного алгоритма умножения матриц, алгоритма Винограда и модифицированного алгоритма Винограда;
		\item[4)] реализовать три алгоритма умножения матриц на одном из языков программирования;
		\item[5)] сравнить алгоритмы умножения матриц.
	\end{enumerate}
	
	\newpage
	\chapter{Аналитическая часть}
	В данном разделе представлено математическое описание алгоритмов умножения матриц.
	
	\section{Стандартный алгоритм умножения матриц}
	Матрица — математический объект, записываемый в виде прямоугольной таблицы элементов кольца или поля (например, целых, действительных или комплексных чисел), которая представляет собой совокупность строк и столбцов, на пересечении которых находятся её элементы. Количество строк и столбцов задает размер матрицы. Хотя исторически рассматривались, например, треугольные матрицы, в настоящее время говорят исключительно о матрицах прямоугольной формы, так как они являются наиболее удобными и общими. 
	Умножение матриц — одна из основных операций над матрицами. Матрица, получаемая в результате операции умножения, называется произведением матриц.
	Пусть даны две прямоугольные матрицы A и B размером $[l * m]$ и $[m * n]$. В результате произведения матриц A и B получим матрицу $C$ размером $[l * n]$, в которой:
	\begin{equation}
		c_{i,j} = \sum_{r=1}^{m}a_{ir}b_{rj} \qquad (i=1,2,...l; j = 1,2,...n)
	\end{equation}
	Операция умножения двух матриц выполнима только в том случае, если число столбцов в первом сомножителе равно числу строк во втором; в этом случае говорят, что матрицы согласованы. В частности, умножение всегда выполнимо, если оба сомножителя — квадратные матрицы одного и того же порядка. 
	
	\section{Алгоритм Винограда}
	Если посмотреть на результат умножения двух матриц, то видно, что каждый элемент в нем представляет собой скалярное произведение соответствующих строки и столбца исходных матриц. Можно заметить также, что такое умножение допускает предварительную обработку, позволяющую часть работы выполнить заранее. \par
	Рассмотрим два вектора $V = (v_{1},v_{2},v_{3},v_{4})$ и $W = (w_{1},w_{2},w_{3},w_{4})$. Их скалярное произведение равно:
	\begin{equation}
		V * W = v_{1}w_{1} + v_{2}w_{2} + v_{3}w_{3} + v_{4}w_{4}
	\end{equation}
	Это равенство можно переписать в виде:
	\begin{equation}
		V * W = (v_{1} + w_{2})(v_{2} + w_{1}) + (v_{3} + w_{4})(v_{4} + w_{3}) -  v_{1}v_{2} - v_{3}v_{4} - w_{1}w_{2} - w_{3}w_{4}
	\end{equation}
	Менее очевидно, что выражение в правой части последнего равенства допускает предварительную обработку: его части можно вычислить заранее и запомнить для каждой строки первой матрицы и для каждого столбца второй. На практике это означает, что над предварительно обработанными элементами придется выполнять лишь первые два умножения и последующие пять сложений, а также дополнительно два сложения. 
	
	\section{Вывод}
	Было представлено математическое описание стандартного алгоритма умножения матриц и алгоритма Винограда. Основное отличие - наличие предварительной обработки, а также количество операций умножения.
	
	\newpage
	\chapter{Конструкторская часть}
	В данном разделе представлены схемы разработанных алгоритмов. Также оценивается трудоемкость алгоритмов.
	\section{Разработка алгоритмов}
	На \hyperref[Algorithm1]{рисунке 1} изображена схема стандартного алгоритма умножения матриц.
	\begin{figure}[h!]\label{Algorithm1}
		\center{\includegraphics[scale=0.9]{source/Algorithm1.png}}
		\caption*{Рисунок 1. Схема стандартного алгоритма умножения матриц}
	\end{figure}
	\newpage
	На \hyperref[Algorithm2]{рисунке 2} изображена схема алгоритма Винограда.
	\begin{figure}[h!]\label{Algorithm2}
		\center{\includegraphics[scale=0.9]{source/Algorithm2.png}}
		\caption*{Рисунок 2. Схема алгоритма Винограда}
	\end{figure}
	\newpage
	На \hyperref[Algorithm3]{рисунке 3} изображена схема модифицированного алгоритма Винограда.
	\begin{figure}[h!]\label{Algorithm3}
		\center{\includegraphics[scale=0.9]{source/Algorithm3.png}}
		\caption*{Рисунок 3. Схема модифицированного алгоритма Винограда}
	\end{figure}
	
	\newpage
	\section{Трудоемкость алгоритмов}
	Модель трудоемкости для оценки алгоритмов:
	\begin{enumerate}
		\item[1)] стоимость базовых операций единица:\par
		$=,+,*,\simeq,<,>,\geq,\leq,==,!=,[],+=,-=,*=,/=,++,--$;
		\item[2)] стоимость цикла:\par
		$f_{for}=f_{init}+f_{comp}+M(f_{body}+f_{increment}+f_{comp})$\par
		Пример: $for(i=0,i<M;i++){/* body */}$\par
		Результат: $2 + M(2+f_{body})$;
		\item[3)] стоимость условного оператора\par
		Пусть goto (переход к одной из ветвей) стоит 0, тогда\par
		\begin{displaymath}
			f_{f} = \left\{ \begin{array}{l l}
				min(f_{A},f_{B}), & \textrm{лучший случай}\\
				max(f_{A},f_{B}), & \textrm{худший случай}\\
			\end{array} \right.
		\end{displaymath}
		\item[4)] операция обращения к ячейки матрицы [i, j] имеет трудоёмкость равную двум.
	\end{enumerate}
	
	\newpage
	литература 
	Виноград: http://www.algolib.narod.ru/Math/Matrix.html
\end{document}

