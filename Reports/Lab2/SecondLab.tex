\documentclass[12pt]{report}
\usepackage[utf8]{inputenc}
\usepackage[english, russian]{babel}
\usepackage{fontspec} 
\usepackage[14pt]{extsizes}
\usepackage{caption}
\usepackage{graphicx}
\usepackage[left=2cm, right=2cm, top=1cm, bottom=2.5cm]{geometry}
\usepackage{listings}
\usepackage{titlesec, blindtext, color} % подключаем нужные пакеты
\usepackage{setspace} % полуторный интервал
\usepackage{hyperref}
\definecolor{gray75}{gray}{0.75} % определяем цвет
\newcommand{\hsp}{\hspace{20pt}} % длина линии в 20pt
% titleformat определяет стиль
\titleformat{\chapter}[hang]{\Huge\bfseries}{\thechapter\hsp\textcolor{gray75}\hsp}{0pt}{\Huge\bfseries}

% Для листинга кода:
\lstset{ 
	language=[Sharp]C,                 % выбор языка для подсветки (здесь это С)
	basicstyle=\small\sffamily, % размер и начертание шрифта для подсветки кода
	numbers=left,               % где поставить нумерацию строк (слева\справа)
	numberstyle=\tiny,           % размер шрифта для номеров строк
	stepnumber=1,                   % размер шага между двумя номерами строк
	numbersep=5pt,                % как далеко отстоят номера строк от подсвечиваемого кода
	showspaces=false,            % показывать или нет пробелы специальными отступами
	showstringspaces=false,      % показывать или нет пробелы в строках
	showtabs=false,             % показывать или нет табуляцию в строках
	frame=single,              % рисовать рамку вокруг кода
	tabsize=2,                 % размер табуляции по умолчанию равен 2 пробелам
	captionpos=t,              % позиция заголовка вверху [t] или внизу [b] 
	breaklines=true,           % автоматически переносить строки (да\нет)
	breakatwhitespace=false, % переносить строки только если есть пробел
}
\defaultfontfeatures{Ligatures={TeX},Renderer=Basic} 
\setmainfont[Ligatures={TeX,Historic}]{Times New Roman}
\begin{document}
	
	%\def\chaptername{} % убирает "Глава"
	\begin{titlepage}
		\begin{table}[ht]
			\centering
			\begin{tabular}{|c|p{420pt}|} 
				\hline
				\begin{tabular}[c]{@{}c@{}} \includegraphics[scale=0.37]{source/EmblemBMSTU} \\\end{tabular} &
				\footnotesize\begin{tabular}[c]{@{}c@{}}\textbf{Министерство~науки~и~высшего~образования~Российской~Федерации}\\\textbf{Федеральное~государственное~бюджетное~образовательное~учреждение}\\\textbf{~высшего~образования}\\\textbf{«Московский~государственный~технический~университет}\\\textbf{имени~Н.Э.~Баумана}\\\textbf{(национальный~исследовательский~университет)»}\\\textbf{(МГТУ~им.~Н.Э.~Баумана)}\\\end{tabular}  \\
				\hline
			\end{tabular}
		\end{table}
		\noindent\rule{\textwidth}{4pt}
		\noindent\rule[14pt]{\textwidth}{1pt}
		\hfill 
		\noindent
		\makebox{ФАКУЛЬТЕТ~}%
		\makebox[\textwidth][l]{\underline{~~~~«Информатика и системы управления»~~~~~~~~~~~~~~~~~~~~~~~~~~~~~~~~~~~~~~~~~~~~}}%
		\\
		\noindent
		\makebox{КАФЕДРА~}%
		\makebox[\textwidth][l]{\underline{~~~~~~~«Программное обеспечение ЭВМ и информационные технологии»~~~~~~~~}}%
		\\
		
		
		\begin{center}
			\vspace{3cm}
			{\bf\huge Отчёт\par}
			{\bf\Large по лабораторной работе № 1\par}
			\vspace{0.5cm}
		\end{center}
		
		
		\noindent
		\makebox{\large{\bf Название:}~~~}
		\makebox[\textwidth][l]{\large\underline{~Расстояния Левенштейна и Дамерау-Левенштейна~}}\\
		
		\noindent
		\makebox{\large{\bf Дисциплина:}~~~}
		\makebox[\textwidth][l]{\large\underline{~Анализ алгоритмов~}}\\
		
		\vspace{1.5cm}
		\noindent
		\begin{tabular}{l c c c c c}
			Студент      & ~ИУ7-55Б~               & \hspace{3.5cm} & \hspace{3.5cm}                 & &  Д.О. Склифасовский \\\cline{2-2}\cline{4-4} \cline{6-6} 
			\hspace{3cm} & {\footnotesize(Группа)} &                & {\footnotesize(Подпись, дата)} & & {\footnotesize(И.О. Фамилия)}
		\end{tabular}
		
		\vspace{1cm}
		
		\noindent
		\begin{tabular}{l c c c c}
			Преподователь & \hspace{6cm}   & \hspace{3.5cm}                 & & ~~~~~~ Л.Л. Волкова ~~~~~~\\\cline{3-3} \cline{5-5} 
			\hspace{3cm}  &                & {\footnotesize(Подпись, дата)} & & {\footnotesize(И.О. Фамилия)}
		\end{tabular}
		
		\begin{center}	
			\vfill
			\large \textit {Москва, 2020}
		\end{center}
		
		\thispagestyle {empty}
		\pagebreak
	\end{titlepage}
	\restoregeometry
	
	\tableofcontents
	\onehalfspacing
	\newpage
	\chapter*{Введение}
	\addcontentsline{toc}{chapter}{Введение}
	
	
\end{document}

